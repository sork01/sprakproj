\documentclass[a4paper,11pt]{article}
\usepackage[margin=2.5cm]{geometry}
\linespread{1.25}

\usepackage[T1]{fontenc}
\usepackage[utf8]{inputenc}
\usepackage[swedish]{babel}
% \usepackage{fontspec}
% \setmainfont{Linux Libertine O}
\usepackage{lmodern}
\usepackage{listings}

\usepackage{fancyhdr}
\pagestyle{fancy}
\renewcommand{\headrulewidth}{ 1.0pt }
\renewcommand{\footrulewidth}{ 0.4pt }
\fancyhead{} % clear all headers
\fancyfoot{} % clear all footers

% E:even page, O:odd page, L:left, C:center, R:right
%\fancyhead[LE,RO]{\thepage}
%\fancyfoot[C]  {Albert Einstein, 2009}

\fancyhead[L]{Meningsåterställare}
\fancyhead[R]{Mikael Forsberg, Robin Gunning}
\fancyfoot[C]{\thepage}

\begin{document}
\renewcommand*{\ttdefault}{pcr}
\lstset{
    escapeinside={(*@}{@*)},
    basicstyle=\footnotesize\ttfamily,
    breaklines=true,
    xleftmargin=0.6cm,
    showstringspaces=false,
    frame=none,
    % numbers=left,
    language=Java,
    inputencoding=utf8,
    extendedchars=true,
    literate={≤}{{$\leq$}}1
}

\lstset{numberstyle=\footnotesize\ttfamily,breaklines=true}

\lstdefinelanguage{algolur}
{
morekeywords={Algorithm,Input,Output,Procedure,and,break,for,while,to,downto,swap,if,or,then,else,do,return,True,False},
sensitive=true,
}

\title{\vspace{-1cm} Meningsåterställare via ord-n-gramstatistik för svenska\\ \small Språkteknologi DD2418 HT2016}
\author{Mikael Forsberg <miforsb@kth.se>\\Robin Gunning <rgunning@kth.se>}
\maketitle

\begin{center}
Abstrakt\\ 
Vi finner att en enkel algoritm presterar bättre än slumpen för att återställa
ordningen på ord i en mening där ordningen gått förlorad.
\end{center}

\section*{Introduktion}
I de flesta meningar i svenska språket är ordningen mellan orden mycket
viktiga för meningens innebörd. En mening där ordningen mellan orden
på något sätt gått förlorad kan ofta se ut som rent nonsens. För en
människa är det dock relativt enkelt att lista ut (inte nödvändigtvis
entydigt) vad som egentligen menas. Desto svårare är att konstruera
ett algoritm som löser uppgiften automatiskt. Denna rapport beskriver ett försök
att konstruera och utvärdera en sådan algoritm. Tänkbara tillämpningar
för algoritmen är exempelvis grammatikkontroll.

\subsection*{Problemformulering}
Givet en mening (en följd svenska ord) där ordningen mellan orden påstås eller
antags vara felaktig, återställ ordningen så gott som möjligt.

\subsection*{Avgränsning}
Per uppgiftslydelse avgränsar vi till att endast använda ord-n-gramstatistik
för svenska för att avgöra vilken ordning som ska anses vara den bästa.

\subsection*{Huvudsaklig angreppsmetod}
Per inrådan från lärare har vi angripit problemet genom att göra en
totalsökning över meningens permutationer och beräkna vilken av dessa
permutationer som har störst sannolikhet att vara korrekt.

\section*{Metod}
\subsection*{Statistikinsamling}
Korpusar hämtades från Språkbanken\textsuperscript{1} vid Göteborgs
Universitet. Vi valde att filtrera bort meningar som innehöll annat än
vanligt språk såsom webbadresser, samt att ersätta ensamma tal (ord
bestående av endast siffror, punkt och komma) med strängen
\texttt{''NUMBER''}. Andra numeriska fragment såsom datum togs med
utan att ersättas med någonting annat.

\vspace{0.25cm}
\noindent
Följande korpusar användes (i olika kombinationer) för statistikinsamling:

\vspace{0.25cm}
% gp2010 (1137633 sentences) (13896 skipped) (425160 words) (3558686 bigrams) (7539724 trigrams)
% gp2011 (1309053 sentences) (16426 skipped) (424646 words) (3536672 bigrams) (7495740 trigrams)
% bloggmix2011 () (411626 words) (3928737 bigrams) (9893028 trigrams)
% wikipedia (9861817 sentences) (238015 skipped) (2590695 words) (20901417 bigrams) (45930124 trigrams)

\begin{center}
\begin{tabular}{l r r r}
\textbf{Korpus}            & \textbf{Unika bigram}   & \textbf{Unika trigram}   &  \textbf{Unika ord} \\
GP2010\textsuperscript{2}                     & 3 558 686                 & 7 539 724                  & 425 160\\
GP2011\textsuperscript{3}                     & 3 536 672                 & 7 495 740                  & 424 646\\
**Bloggmix 2011\textsuperscript{4}              & 3 928 737                 & 9 893 028                  & 411 626\\
Svenska Wikipedia-korpusen\textsuperscript{5} & 20 901 417                & 45 930 124 
\end{tabular}
\end{center}

\hfill {\footnotesize* Vi lyckades endast parsa en mindre del av Bloggmix2011 (libxml2 användes)}

\subsection*{Algoritm}
Översiktligt fungerar algoritmen enligt följande:

\begin{enumerate}
\item Läs in en sträng (meningen) och dela upp den i enstaka ord. Ersätt ensamma tal med strängen ''NUMBER''.
\item För varje permutation av listan på meningens ord, konstruera
alla (ord-) bi- och trigram av permutationen och räkna hur
många av dessa som inte finns i statistiken. Placera de
permutationer som har lägst antal n-gram som inte täcks
av statistiken i en lista av kandidatpermutationer.
\item För varje permutation i listan över kandidatpermutationer,
summera de statistiska frekvenserna av dess bi- och trigram
med hänsyn till statistiken över vilka bi- och trigram
som stått som meningsbörjan eller -slut. Vikter kan införas och justeras
genom utvärdering.
Välj den permutation som ger högst summa som slutgiltigt svar. Om
flera permutationer får samma summa, välj den sista av dessa.
\end{enumerate}

Pseudokod:

\begin{lstlisting}[language=algolur]
Input: (string) scrambledSentence
          (map) bigramFreqStart
          (map) bigramFreqMid
          (map) bigramFreqEnd
          (map) trigramFreqStart
          (map) trigramFreqMid
          (map) trigramFreqEnd

Output: prints corrected sentence

Procedure:
    scrambledSentence (*@$\leftarrow$@*) replace(/ [\d.,]+ /, " NUMBER ", scrambledSentence)
    words (*@$\leftarrow$@*) split(scrambledSentence, " ") // split by SPACEs
    candidates (*@$\leftarrow$@*) list()
    leastZeroes (*@$\leftarrow$@*) length(words) (*@$\cdot$@*) 1000;
    
    for x (*@$\in$@*) permutations(words):
        zeroes (*@$\leftarrow$@*) 0
        
        for y (*@$\in$@*) bigrams(words), s (*@$\in$@*) (bigramFreqStart or bigramFreqMid or bigramFreqEnd):
            if y (*@$\not \in$@*) s: zeroes (*@$\leftarrow$@*) zeroes + 1
        for y (*@$\in$@*) trigrams(words), s (*@$\in$@*) (trigramFreqStart or trigramFreqMid or trigramFreqEnd):
            if y (*@$\not \in$@*) s: zeroes (*@$\leftarrow$@*) zeroes + 1
        
        if zeroes < leastZeroes:
            leastZeroes (*@$\leftarrow$@*) zeroes
            candidates.clear()
            candidates.add(x)
        else if zeroes = leastZeroes:
            candidates.add(x)
    
    answer (*@$\leftarrow$@*) ""
    bestScore (*@$\leftarrow$@*) 0
    
    for x (*@$\in$@*) candidates(words):
        score (*@$\leftarrow$@*) 0
        
        score (*@$\leftarrow$@*) score + bigramFreqStart[first(bigrams(word))] (*@$\cdot$@*) 2
        score (*@$\leftarrow$@*) score + bigramFreqEnd[last(bigrams(word))] (*@$\cdot$@*) 2
        
        score (*@$\leftarrow$@*) score + trigramFreqStart[first(trigrams(word))] (*@$\cdot$@*) 2
        score (*@$\leftarrow$@*) score + trigramFreqEnd[last(trigrams(word))] (*@$\cdot$@*) 2
        
        for y (*@$\in$@*) middle(bigrams(words)):
            score (*@$\leftarrow$@*) score + bigramFreqMid[y]
        for y (*@$\in$@*) middle(trigrams(words)):
            score (*@$\leftarrow$@*) score + trigramFreqMid[y] (*@$\cdot$@*) 3
        
        if score (*@$\geq$@*) bestScore:
            bestScore (*@$\leftarrow$@*) score
            answer (*@$\leftarrow$@*) join(x, " ")
    
    print(answer)
\end{lstlisting}

\subsection*{Utvärdering}
För att utvärdera algoritmen definierades ett korrekthetsmått mellan
en given korrekt mening och den av algoritmen föreslagna meningen.
Måttet ges av andelen meningsdelar (ord-n-gram)
av storlek två eller fler ord i den föreslagna meningen som återfinns
i den korrekta meningen. Beräkningen illustreras enklast med ett exempel
där de meningsdelar som återfinns i den korrekta meningen markerats
med fetstil:

\begin{quote}
Korrekt = ''vi håller \textbf{en presentation}''

Föreslagen = ''vi \textbf{en presentation} håller''

Meningsdelar i förslaget: ''vi en'', ''\textbf{en presentation}'', ''presentation håller''

\hspace{4.2cm} ''vi en presentation'', ''en presentation håller''

\hspace{4.2cm} ''vi en presentation håller''

Korrekthet: $\displaystyle \frac{1}{6}$
\end{quote}

\vspace{0.25cm}
\noindent
En uppsättning testmeningar
skapades utifrån Språkbankens korpus LäSBarT\textsuperscript{6}. Testmeningarna
grupperades efter antal ord.

\begin{center}
\begin{tabular}{l l}
\textbf{Antal ord} & \textbf{Antal testmeningar}\\
3 & 1561\\
4 & 1627\\
5 & 1549\\
6 & 1279\\
7 & 1028\\
8 & 720\\
\end{tabular}
\end{center}

\noindent
Algoritmens prestanda jämfördes med att slumpa ordningen på orden med \texttt{random.shuffle}
i Python 3.5.2 (Linux 4.7, 64-bit), vilket fick fungera som en slags nollhypotes.

\vspace{0.25cm}
\noindent

\section*{Resultat}
\subsection*{Bästafall: statistik från LäSBarT}
Detta resultat ger inte ett mått på hur väl algoritmen fungerar i allmänhet, utan
ger ett slags tak för hur bra algoritmen kan prestera i bästa tänkbara fall.

\vspace{0.25cm}
\noindent
Tabellen visar aritmetiska medelvärden för korrekthetsmåttet över samtliga testmeningar.

\begin{center}
\begin{tabular}{l l l}
\textbf{Antal ord i mening} & \textbf{Meningsåterställare} & \textbf{\texttt{random.shuffle}}\\
3 & 0.985 & 0.270\\
4 & 0.986 & 0.172\\
5 & 0.979 & 0.103\\
6 & 0.989 & 0.067\\
7 & 0.980 & 0.051\\
8 & 0.987 & 0.042\\
\end{tabular}
\end{center}

\subsection*{Statistik från GP2010}

Tabellen visar aritmetiska medelvärden för korrekthetsmåttet över samtliga testmeningar.
\begin{center}
\begin{tabular}{l l l}
\textbf{Antal ord i mening} & \textbf{Meningsåterställare} & \textbf{\texttt{random.shuffle}}\\
3 & 0.545 & 0.294\\
4 & 0.507 & 0.173\\
5 & 0.437 & 0.105\\
6 & 0.372 & 0.072\\
7 & 0.303 & 0.056\\
8 & 0.255 & 0.039\\
\end{tabular}
\end{center}

\subsection*{Statistik från Svenska Wikipedia-korpusen}

Tabellen visar aritmetiska medelvärden för korrekthetsmåttet över samtliga testmeningar.
\begin{center}
\begin{tabular}{l l l}
\textbf{Antal ord i mening} & \textbf{Meningsåterställare} & \textbf{\texttt{random.shuffle}}\\
3 & 0.464 & 0.296\\
4 & 0.444 & 0.167\\
5 & 0.395 & 0.100\\
6 & 0.323 & 0.070\\
7 & 0.273 & 0.051\\
8 & 0.237 & 0.039\\
\end{tabular}
\end{center}

\subsection*{Statistik från sammanslagning av korpusar}

För detta test slog vi ihop korpusarna Bloggmix2011, GP2010, GP2011, och Svenska
Wikipedia-korpusen.

\vspace{0.25cm}
\noindent
Tabellen visar aritmetiska medelvärden för korrekthetsmåttet över samtliga testmeningar.
\begin{center}
\begin{tabular}{l l l}
\textbf{Antal ord i mening} & \textbf{Meningsåterställare} & \textbf{\texttt{random.shuffle}}\\
3 & 0.629 & 0.290\\
4 & 0.540 & 0.157\\
5 & 0.457 & 0.104\\
6 & 0.374 & 0.074\\
7 & 0.308 & 0.050\\
8 & 0.266 & 0.042\\
\end{tabular}
\end{center}

\section*{Diskussion}
\subsection*{Algoritmens prestanda}
Algoritmen presterar enligt testerna bättre än att använda \texttt{random.shuffle} för att slumpa fram
ordningen mellan orden. Sett till det egna korrekthetsmåttet presterar algoritmen bättre på
kortare meningar, men sett relativt mot \texttt{random.shuffle} presterar algoritmen betydligt
bättre ju längre meningarna blir. En större korpus för statistik ger som väntat bättre resultat, men
en större korpus av en språktyp som inte matchar språktypen i indata ger sämre resultat, vilket
ses i testet med statistik baserad på Svenska Wikipedia-korpusen.

\subsection*{Korrekthetsmåttet}
Korrekthetsmåttet är strängt i sin bedömning, särskilt i sådana fall där
algoritmen föreslår en helt ekvivalent mening.

\begin{quote}
\texttt{Korrekt: ''det har \textbf{ingen annan} klarat''}

\texttt{Förslag: ''\textbf{ingen annan} har klarat det''}

\texttt{Score: 0.10}
\end{quote}

\vspace{0.25cm}
\noindent
Att detektera denna typ av ekvivalens skulle dock antagligen kräva en semantisk parser

\subsection*{Förbättringsförslag}
Det som först och främst bör förbättras är utvärderingen. Denna kan enkelt göras betydligt
mer omfattande, exempelvis genom att inkludera många fler uppsättningar testmeningar,
gärna från olika förberäknade kategorier såsom nyhetstexter, skönlitterära texter och
dialogtexter. Med en större testmängd kan man sedan försöka förbättra algoritmen
genom att exempelvis justera vikterna och utvärdera. Det verkar även troligt att 
algoritmen skulle prestera bättre givet ännu större statistikbaser samt med statistik
som omfattar n-gram av ordning större än 3. Att utöka statistiken (särskilt genom att
inkludera större n-gram) innebär dock betydligt ökade krav på diskutrymme, vilket
skulle göra ett program som använder algoritmen mindre praktiskt.

\section*{Slutsats}
En enkel totalsökningsalgoritm som använder ord-n-gramstatistik över svenska för
att givet en mening där ordningen mellan ord gått förlorad rangordna vilken permutation
av ordmängden som är mest sannolik att vara ''korrekt'' ger bättre resultat än att slumpa
fram en permutation.

\section*{Referenser}

\begin{tabular}{r l}
1 & \small \texttt{https://spraakbanken.gu.se/}\\
2 & \small \texttt{https://spraakbanken.gu.se/resource/gp2010}\\
3 & \small \texttt{https://spraakbanken.gu.se/resource/gp2011}\\
4 & \small \texttt{https://spraakbanken.gu.se/resource/bloggmix2011}\\
5 & \small \texttt{https://spraakbanken.gu.se/eng/resource/wikipedia-sv}\\
6 & \small \texttt{https://spraakbanken.gu.se/resource/lasbart}\\
\end{tabular}
\end{document}
